\documentclass[12pt]{article}

%% FONTS
%% To get the default sans serif font in latex, uncomment following line:
 \renewcommand*\familydefault{\sfdefault}
%%
%% to get Arial font as the sans serif font, uncomment following line:
%% \renewcommand{\sfdefault}{phv} % phv is the Arial font
%%
%% to get Helvetica font as the sans serif font, uncomment following line:
% \usepackage{helvet}
\usepackage[small,bf,up]{caption}
\renewcommand{\captionfont}{\footnotesize}
\usepackage[left=1in,right=1in,top=1in,bottom=1in]{geometry}
\usepackage{graphics,epsfig,graphicx,float,subfigure,color}
\usepackage{amsmath,amssymb,amsbsy,amsfonts,amsthm}
\usepackage{url}
\usepackage{boxedminipage}
\usepackage[sf,bf,tiny]{titlesec}
 \usepackage[plainpages=false, colorlinks=true,
   citecolor=blue, filecolor=blue, linkcolor=blue,
   urlcolor=blue]{hyperref}
\usepackage{enumitem}
\usepackage{verbatim}
\usepackage{tikz,pgfplots}

\newcommand{\todo}[1]{\textcolor{red}{#1}}
% see documentation for titlesec package
% \titleformat{\section}{\large \sffamily \bfseries}
\titlelabel{\thetitle.\,\,\,}

\newcommand{\bs}{\boldsymbol}
\newcommand{\alert}[1]{\textcolor{red}{#1}}
\setlength{\emergencystretch}{20pt}

\begin{document}

\begin{center}
  \vspace*{-2cm}
{\small MATH-GA 2012.001 and CSCI-GA 2945.001, Georg Stadler \&
  Dhairya Malhotra (NYU Courant)}
\end{center}
\vspace*{.5cm}
\begin{center}
\large \textbf{%%
Spring 2019: Advanced Topics in Numerical Analysis: \\
High Performance Computing \\
Assignment 5 (due Apr.\ 29, 2019) }\\
Chen Li cl3898
\end{center}

\noindent {\bf Handing in your homework:} Hand in your homework as for
the previous homework assignments (git repo with Makefile), answering
the questions by adding a text or a \LaTeX\ file to your repo.
\\[.2ex]

% ****************************
\begin{enumerate}
% --------------------------

\item {\bf MPI ring communication.}  Write a distributed memory
  program that sends an integer in a ring starting from process 0 to 1
  to 2 (and so on). The last process sends the message back to process
  0. Perform this loop $N$ times, where $N$ is set in the program or
  on the command line.
  \begin{itemize}
  \item Start with sending the integer 0 and let every process add its
    rank to the integer it received before it sends it to the next
    processor. Use the result
    after $N$ loops to check if all processors have properly added
    their contribution each time they received and sent the message.
  \item Time your program for a larger $N$ and estimate the latency on
    your system (i.e., the time used for each communication).  If you
    have access to the CIMS network, try to test your communication
    ring on more than one machine such that communication must go
    through the network.\footnote{See the computing \@ CIMS
      information sheet posted in the first week of the class
      on how to use multiple hosts when using \texttt{mpirun}. Note
      that on each host, the same compiled file must be available,
      i.e., they need to have a shared directory such as the home
      directories in CIMS.} If you
    use MPI on a single processor with multiple cores, the available
    memory is logically distributed, but messages are not actually
    sent through a network.\footnote{It depends on the implementation
      of MPI how it handles sending/receiving of data that is stored
      in the same physical memory.}
  \item Hand in your solution using the filename \texttt{int\_ring.c}.
  \item Modify your code such that instead of a single integer being
    sent in a ring, you communicate a large array of about 2MByte in a
    ring. Time the communication and use these timings to estimate the
    bandwidth of your system (i.e., the amount of data that can be
    communicated per second).
  \end{itemize}
  
	\textbf{Solution}
	Please see the int\_ring.c for code. And I ran it on CIMS machine with different processes. 
	
	\begin{table}[h!]
\centering
\begin{tabular}{ |c|c|c|c|c|c| }
\hline
Processes&	Iterations&	Time &	Time of Array&	Latency&	Bandwidth (GB/s)\\
\hline
10&	1&	0.003754&	0.073924&	0.000412&	0.023732\\
\hline
30&	1&	0.052108&	0.286345&	0.001763&	0.006452\\
\hline
10&	100&	0.147934&	8.139241&	0.000176&	0.000224\\
\hline
30&	100&	0.727354&	54.63829&	0.000245&	0.000028\\
\hline
2&	1000&	0.353984&	39.287925&	0.000174&	0.000039\\
\hline
10&	1000&	0.603281&	78.973921&	0.000061&	0.000020\\
\hline
\end{tabular}
 \caption{Performance on crunchy5 and crunchy6}
 \label{mpi1}
 \end{table} 


\item {\bf Provide details regarding your final project.}
  Provide a more detailed list/table of what tasks you have done and are planning on
  doing for your final project, including an estimate of the  week
  you will be working on each task, and who will work on it. We are
  thinking of something like this:
  \begin{center}
  \begin{tabular} {|c|p{9cm}|p{2cm}|}
    \hline
    \multicolumn{3}{|c|}{\bf Project: Image Denoising with Total Variation in GPU} \\
    \hline
    Week & Work & Who  \\ \hline \hline
    04/20-04/25 & Read potential paper and algorithm & Chen, Shengqi \\ \hline
    04/26-05/02 & Write pseudo code for the corresponding algorithm  & Chen, Shengqi \\ \hline
    05/02-05/05 &  Implement basic code in C & Chen, Shengqi\\ \hline
    05/06-05/15 & Covert the code into GPU version and debugging & Richard, Erlich \\ \hline
    05/16-05/19 & Run some test for the codes  & Chen, Shengqi \\ \hline
  \end{tabular}
  \end{center}



\end{enumerate}
\end{document}
